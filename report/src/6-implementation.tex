\section{Implementazione}\label{sec:implementazione}

In questa sezione verranno trattati gli aspetti inerenti all’implementazione hardware
e software legata allo sviluppo del sistema: in primo luogo saranno velocemente
presentate le tecnologie utilizzate all’interno di una o più delle entità in
gioco nel sistema, successivamente verrà dedicata una sottosezione a ciascuna di
queste dove saranno analizzate nello specifico la sua architettura e le funzionalità
di dettaglio.

\subsection{Tecnologie Utilizzate}

Di seguito sono riportate in un elenco le tecnologie più rilevanti utilizzate per l'implementazione del progetto.

\begin{description}
  \item[Typescript \& Angular: ]
    Per la realizzazione dei frontend web del server cloud e del server edge, è stato impiegato il framework Angular con il linguaggio consigliato Typescript.
    \textbf{Typescript} è un Super-set di JavaScript ES6 e, come tale, è un linguaggio di scripting orientato agli oggetti e agli eventi; Angular, invece, è un framework open source per lo sviluppo di applicazioni web con licenza MIT, evoluzione di AngularJS e sviluppato principalmente da Google.
  \item[Cloud Firestore: ]
    Per memorizzare i dati raccolti è stato utilizzato \textbf{Cloud Firestore}, un database flessibile e scalabile per lo sviluppo di dispositivi mobili, Web e server da Firebase e Google Cloud Platform.
  \item[Kotlin: ]
    Per l'implementazione di backend e gestione delle API è stato utilizzato \textbf{Kotlin}, un linguaggio di programmazione general purpose, multi-paradigma (object-oriented e funzionale), open source sviluppato dall'azienda di software JetBrains e basato su JVM.
  \item[Vertx: ]
  Libreria utilizzata all’interno dei componenti che si basano su una JVM, mette a disposizione una serie di primitive e strutture che non solo consentono di modellare un’architettura asincrona e basata sullo scambio di messaggi all’interno del singolo componente, ma permette una gestione semplice, veloce e immediata delle chiamate REST sia lato client che server.
  Tra gli altri si è optato per Vert.x, oltre che per la sua struttura basata sugli eventi asincroni all’interno di un event-loop, per la sua semplicità di deploy. A differenza di altri web server più conosciuti all’interno del mondo JVM come Java EE o Spring, Vert.x è totalmente indipendente e non necessita di un servlet container, permettendo perciò un deploy ancora più immediato.
\end{description}

\subsection{Backend Cloud}

\subsection{Frontend Cloud e Fog}

Lo sviluppo dell'interfaccia web dell'applicativo è stato realizzato attraverso il framework Angular utilizzando il linguaggio Typescript. Si è proceduto principalmente per step successivi, cercando di realizzare inizialmente uno scheletro iniziale dei vari componenti web, per poi andare a migliorare aggiungendo dettagli e features di volta in volta.

  \subsubsection{Frontend Cloud}

  L'interfaccia web per la parte Cloud consiste in una rappresentazione dei dati attraverso due grafici, entrambi realizzati grazie alla libreria \texttt{Chart.js}: uno a linee ed uno a torta.
  Il grafico a linee rappresenta l'andamento nel tempo (nell'intervallo prestabilito di 2 ore) dell'affollamento rilevato per tutti i potenziali fog (ognuno rappresentato da una linea); sull'asse delle ascisse il tempo, su quello delle ordinate il quantitativo di dispositivi trovati.
  Il grafico a torta viene utilizzato per mostrare una statistica relativa ai vari vendor (ricavabili dal MAC address) di ogni dispositivo rilevato.

  \subsubsection{Frontend Fog}

  L'interfaccia web per la parte Fog consiste in una rappresentazione tabellare dei dispositivi rilevati (comprendente vendor e un hash del MAC address per ogni dispositivo), un grafico a linee concettualmente simile a quello sviluppato nel frontend cloud ed un contatore del totale dei dispositivi trovati.

\subsection{Nodi fog}

\subsection{Sensori IoT}
