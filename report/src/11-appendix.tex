\appendix

\renewcommand{\thesection}{\Alph{section}}
\renewcommand{\thesubsection}{A.\arabic{subsection}}

\addcontentsline{toc}{section}{Appendice}
\section*{Appendice}

% Laddove necessario è possibile avvalersi di appendici alla relazione per includere materiale di approfondimento.\\

% A titolo esemplificativo possono essere incluse le schede tecniche dei componenti adottati, la normativa di riferimento che regola un particolare dominio applicativo, ecc.

\subsection{Configurazione del Raspberry Pi}\label{app:raspi}

Si è scelto di utilizzare un modulo WiFi USB in aggiunta a quello integrato nel Raspberry Pi 3,
ma è possibile utilizzare il solo modulo WiFi integrato sia come access point che come client, pur con maggiori instabilità e minori velocità.

La procedura di adattamento dell'immagine è stata documentata in modo da poter essere riproducibile in futuro.

\begin{enumerate}
  \item
    La base dell'immagine è l'ultima versione di \textbf{Raspbian} nella sua versione minimale \textbf{lite}:
    in questo caso, la versione Buster del 26 settembre 2019.

    Dopo l'installazione, sono stati aggiornati i pacchetti tramite \texttt{apt} via rete WiFi;
  \item
    seguendo la documentazione fornita sul sito ufficiale\footnote{\url{https://raspap.com/}},
    è stato installato \textbf{RaspAP} e configurato per funzionare esclusivamente via WiFi:
    \begin{enumerate}
      \item
        \texttt{curl -sL https://install.raspap.com | bash} permette di installare RaspAP\@;
      \item
        seguendo le FAQ\footnote{\url{https://github.com/billz/raspap-webgui/wiki/FAQs\#can-i-use-wlan0-and-wlan1-rather-than-eth0-for-my-ap}}
        al file \texttt{includes/config.php} si è aggiunta \texttt{wlan1} come client interface e si è impostato nel file \texttt{/etc/dhcpcd.conf}
        \texttt{wlan0} come access point
    \end{enumerate}
    In questo modo, il dispositivo dovrebbe essere in grado di collegarsi alla rete WiFi pre-configurata e mettere a disposizione una rete con SSID ``raspi-webgui''.
  \item
    utilizzando lo script ufficiale\footnote{\url{https://github.com/docker/docker-install}} si è installato Docker;
  \item
    si è installato Java 1.8 tramite pacchetto \texttt{openjdk-8-jdk-headless};
  \item
    il broker MQTT Mosquitto viene eseguito tramite container Docker \texttt{eclipse-mosquitto}.
\end{enumerate}
