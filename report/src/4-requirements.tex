\section{Analisi dei requisiti}

% Esporre brevemente i requisiti a cui il sistema proposto deve rispondere.
% Concentrare l'attenzione sugli aspetti più rilevanti, facendo eventualmente uso di opportuni diagrammi di alto livello.

% Vincoli circa la lunghezza della sezione (escluse didascalie, tabelle, testo nelle immagini, schemi):
% Numero minimo di battute per 2 componenti: 6000
% Numero massimo di battute per 2 componenti: 8000

Per questo progetto, l'obiettivo di massima è la realizzazione di un sistema impiegabile in un ambiente indoor di tipo universitario (in particolare, una biblioteca dotata di aule studio),
monitorando il numero di persone stimato per stanza per tutto il periodo di funzionamento del sistema.
% TODO: ^ non mi convince, meglio rendere più generico e spiegare dopo
% TODO: spiegare che usa il WiFi

Un servizio di monitoraggio dell'affollamento di ambienti indoor consiste in una serie di sensori distribuiti nelle stanze dell'edificio che registrano le presenze;
poiché si assume essere un luogo pubblico, il sistema deve poter monitorare l'ambiente senza dipendere dalla collaborazione attiva da parte dell'utente.

Il servizio deve essere facilmente fruibile e non deve invasivo né per l'edificio né per l'infrastruttura di rete esistente.
I sensori possono dunque essere più o meno complessi compatibilimente con le esigenze dell'ambiente di installazione.

Infine, dai casi riportati in letteratura si può dedurre che il problema della privacy nella gestione dei dati è di fondamentale importanza e va considerato assolutamente.

\subsection{Raccolta dei requisiti}

Per poter condurre un lavoro efficace, si è deciso in inziare dalla scelta di un ambiente reale e di effettuarne l'analisi, eventualmente con la collaborazione della struttura stessa.


\subsection{Requisti funzionali}

\subsection{Requisiti non funzionali}
