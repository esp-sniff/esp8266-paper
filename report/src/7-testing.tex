\section{Testing e performance}

% Esporre lo stato di funzionamento effettivo del sistema progettato ad elaborato concluso.
% Per ciascuna delle funzionalità salienti devono essere tabellate e discusse le performance riscontrate mediante opportuni test eseguiti in fase di validazione del progetto.

% I tempi di esecuzione/comunicazione devono essere accompagnati dalle caratteristiche dell'hardware sul quale è eseguito il software.

% Qualora l'elaborato includa algoritmi innovativi, indicarne la complessità computazionale (avendo cura di esporre lo pseudo codice nella sezione implementazione).

% Vincoli circa la lunghezza della sezione (escluse didascalie, tabelle, testo nelle immagini, schemi):
% Numero minimo di battute per 2 componenti: 2500
% Numero massimo di battute per 2 componenti: 4500

\subsection{Testing durante lo sviluppo}

Per quanto riguarda i piani di test, abbiamo deciso di adottare un approccio manuale:
non potendo contare su particolari framework dedicati, abbiamo realizzato la maggior parte delle prove manualmente.

\subsubsection[Backend]{Backend (Fog \& Cloud)}

Parlando della parte relativa ai backend (sia Fog che Cloud)
web, abbiamo realizzato un testing via Postman[22], strumento specifico per le
API REST: ogni volta che una nuova API veniva aggiunta al sistema, questa veniva
testata con tale strumento per verificare non solo che il server la esponesse corret-
tamente, ma anche che la Buisness Logic a essa collegata corrispondesse a quanto
pensato in fase di progettazione. Parlando delle componenti non relative al web
invece, queste sono state testate mediante un approccio incrementale: ogni com-
ponente è stato verificato nel proprio funzionamento prima di essere integrato con
il resto del sistema, purtroppo molti di questi test hanno richiesto con figurazioni
hardware e software non banali(ad esempio, per testare la comunicazione seriale
tra Raspberry e Arduino si è reso necessario scrivere parecchio codice dedicato e
impostare una specifica logica di ricezione e invio dei messaggi), di conseguenza
tali test sono stati effettuati, ma non inclusi nella release finale degli elaborati.

\subsection{Testing del sistema finale}

\subsubsection{Hardware utilizzato}

Per quanto riguarda l'hardware impiegato per il testing, sono state utilizzate le seguenti risorse:

\begin{itemize}
  \item \textbf{Raspberry Pi 3 Model B} come edge server\todo[inline]{continua} % TODO
\end{itemize}

\subsubsection{Performance ottenute}
